% tuc_qct_paper.tex
% Autor: Anónimo (firma interna: k19)
% Proyecto TUC/QCT — Teoría Unificada de la Coherencia

\documentclass[12pt]{article}
\usepackage[utf8]{inputenc}
\usepackage{amsmath, amssymb}
\usepackage{graphicx}
\usepackage{geometry}
\usepackage{hyperref}
\geometry{margin=1in}

\title{Teoría Unificada de la Coherencia (TUC/QCT)}
\author{Anónimo\footnote{Firma interna: k19}}
\date{\today}

\begin{document}

\maketitle

\begin{abstract}
La \textbf{Teoría Unificada de la Coherencia (TUC/QCT)} propone un marco conceptual y matemático que conecta la coherencia cuántica, el Cero Coherente y el tiempo emergente. Este trabajo presenta la base conceptual, derivaciones matemáticas, ejemplos didácticos y simulaciones computacionales que permiten explorar patrones de coherencia y periodicidad en sistemas universales.
\end{abstract}

\tableofcontents
\newpage

%----------------------------------
\section{Introducción}
La coherencia es una medida de la sincronización de fases en sistemas cuánticos y macroscópicos.  
El \textbf{Cero Coherente} representa un estado fundamental sin tiempo ni movimiento externo, funcionando como el fondo del universo.  
El \textbf{Tiempo Emergente} surge de los cambios de coherencia, permitiendo una temporalidad derivada y no fundamental.

Este documento integra:

\begin{itemize}
    \item Conceptos físicos (coherencia, cero coherente, tiempo emergente, periodicidad)
    \item Ejemplos didácticos para niños y estudiantes
    \item Derivación paso a paso del formalismo Born-Oppenheimer aplicado al tiempo emergente
    \item Resultados de simulaciones computacionales con visualizaciones
\end{itemize}

%----------------------------------
\section{Coherencia y Cero Coherente}

\subsection{Coherencia}
\[
\xi = \frac{1}{\sigma_{\text{phase}}}
\]

- Alta coherencia $\rightarrow$ fases alineadas  
- Baja coherencia $\rightarrow$ fases desordenadas

\begin{verbatim}
Alta coherencia:
* →  * →  * →  * →

Baja coherencia:
* →  * ↑  * ↖  * ↓
\end{verbatim}

% Placeholder figura coherencia
\begin{figure}[h!]
\centering
\includegraphics[width=0.7\textwidth]{../paper/figures/coherence_example.png}
\caption{Ejemplo visual de coherencia (ASCII transformado a PNG)}
\end{figure}

\subsection{Cero Coherente}

- Estado fundamental, todas las partículas alineadas
- Base para derivación BO y simulaciones

\begin{verbatim}
* →  * →  * →  * →  * →  * →
\end{verbatim}

% Placeholder figura Cero Coherente
\begin{figure}[h!]
\centering
\includegraphics[width=0.7\textwidth]{../paper/figures/zero_coherent.png}
\caption{Cero Coherente: estado de coherencia máxima}
\end{figure}

%----------------------------------
\section{Tiempo Emergente}
\[
i \hbar \frac{\partial \psi_{\text{slow}}}{\partial t} = H_{\text{eff}} \psi_{\text{slow}}
\]

Ejemplo ASCII:

\begin{verbatim}
Paso 0: * →  * →  * ↑  * ↖
Paso 1: * →  * →  * →  * ↖   -> t emergente +1
Paso 2: * →  * →  * →  * →   -> coherencia máxima, t estable
\end{verbatim}

% Placeholder figura Tiempo Emergente
\begin{figure}[h!]
\centering
\includegraphics[width=0.7\textwidth]{../paper/figures/emergent_time.png}
\caption{Evolución de tiempo emergente vs coherencia}
\end{figure}

%----------------------------------
\section{Periodicidad}
\[
C(\Delta t) = \frac{\langle \psi(t) \cdot \psi(t+\Delta t) \rangle}{||\psi(t)|| \, ||\psi(t+\Delta t)||}
\]

Ejemplo ASCII:

\begin{verbatim}
Paso 0: * →  * →  * ↑  * ↖
Paso 1: * →  * ↑  * →  * ↗
Paso 2: * →  * →  * ↑  * ↖
\end{verbatim}

% Placeholder figura Periodicidad
\begin{figure}[h!]
\centering
\includegraphics[width=0.7\textwidth]{../paper/figures/periodicity.png}
\caption{Correlación de coherencia mostrando periodicidad ~3.4 años}
\end{figure}

%----------------------------------
\section{Derivación BO y Tiempo Emergente}

\subsection{Separación BO}
\[
\Psi(x, y) = \chi_{\text{fast}}(x; y) \, \psi_{\text{slow}}(y)
\]

Promediando sobre partículas rápidas:

\[
\langle \chi_{\text{fast}} | H | \chi_{\text{fast}} \rangle \psi_{\text{slow}} = E_{\text{eff}} \psi_{\text{slow}}
\]

\subsection{Tiempo Emergente}
\[
i \hbar \frac{\partial \psi_{\text{slow}}}{\partial t} = H_{\text{eff}} \psi_{\text{slow}}
\]

ASCII conceptual:

\begin{verbatim}
Fondo Cero Coherente:
* →  * →  * →  * →
Partículas rápidas:
* ↑  * ↖  * →  * ↗
\end{verbatim}

% Placeholder figura BO
\begin{figure}[h!]
\centering
\includegraphics[width=0.7\textwidth]{../paper/figures/BO_derivation.png}
\caption{Derivación BO y evolución de tiempo emergente}
\end{figure}

%----------------------------------
\section{Simulaciones Computacionales}

- Scripts: `simulation_runner.py`, `periodicity_verification.py`  
- Visualizaciones TXT + PNG  
- Carpetas de resultados organizadas por experimento

\begin{verbatim}
1. Inicializar Cero Coherente
2. Evolución paso a paso
3. Guardar ASCII + PNG
4. Calcular correlación y resumen
\end{verbatim}

% Placeholder figura simulación
\begin{figure}[h!]
\centering
\includegraphics[width=0.7\textwidth]{../paper/figures/simulation_example.png}
\caption{Simulación de evolución de coherencia y tiempo emergente}
\end{figure}

%----------------------------------
\section{Ejemplos y Tutoriales}

- Tutoriales para niños: `coherence_story.md`, `zero_coherent_game.md`  
- Tutoriales avanzados: `introduction_to_coherence.md`, `BO_derivation_step_by_step.md`  

%----------------------------------
\section{Conclusiones y Predicciones}

- La TUC/QCT conecta coherencia ↔ tiempo emergente ↔ periodicidad  
- Predicciones verificables en simulaciones y experimentos futuros  
- Base para juegos educativos y proyectos científicos avanzados

\end{document}
